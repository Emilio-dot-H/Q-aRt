\documentclass[12pt, titlepage]{article}

\usepackage{fullpage}
\usepackage[round]{natbib}
\usepackage{multirow}
\usepackage{booktabs}
\usepackage{tabularx}
\usepackage{graphicx}
\usepackage{float}
\usepackage{hyperref}
\hypersetup{
    colorlinks,
    citecolor=black,
    filecolor=black,
    linkcolor=red,
    urlcolor=blue
}
\usepackage[round]{natbib}

\newcounter{acnum}
\newcommand{\actheacnum}{AC\theacnum}
\newcommand{\acref}[1]{AC\ref{#1}}

\newcounter{ucnum}
\newcommand{\uctheucnum}{UC\theucnum}
\newcommand{\uref}[1]{UC\ref{#1}}

\newcounter{mnum}
\newcommand{\mthemnum}{M\themnum}
\newcommand{\mref}[1]{M\ref{#1}}

\title{SE 3XA3: Software Requirements Specification\\Title of Project}

\author{Team 05, Q-aRt QTs 
\\Elton Schiott schiotek 
\\Emilio Hajj hajje 
\\Liam Duncan duncanla
}

\date{\today}

\begin{document}

\maketitle

\pagenumbering{roman}
\tableofcontents
\listoftables
\listoffigures

\begin{table}[bp]
\caption{\bf Revision History}
\begin{tabularx}{\textwidth}{p{3cm}p{2cm}X}
\toprule {\bf Date} & {\bf Version} & {\bf Notes}\\
\midrule
08/11/16 & 1.0 & Basic design information\\
\bottomrule
\end{tabularx}
\end{table}

\newpage

\pagenumbering{arabic}

\section{Anticipated and Unlikely Changes} \label{SecChange}


\subsection{Anticipated Changes} \label{SecAchange}


\begin{description}
\item[\refstepcounter{acnum} \actheacnum \label{acInputImage}:] 
The range of allowed image sizes and image aspect ratios for the input image is expected to change after testing experimenting with different sizes after revision 0.

\item[\refstepcounter{acnum} \actheacnum \label{QRCodeSizes}:] 
The sizes and selection of levels (number of chars encoded as well as data robustness level) of output QaRt Codes are expected to change with experimentation after revision 0 to deliver the most reliable most convenient elements.

\end{description}

\subsection{Unlikely Changes} \label{SecUchange}


\begin{description}
\item[\refstepcounter{ucnum} \uctheucnum \label{ucAlgorithms}:] 
The general QR encoding algorithms themselves are not likely to see changes. Therefor within reason parts of the process can be spread across multiple modules with low coupling but which all are aware of the math behind other modules which are part of the process.

\item[\refstepcounter{ucnum} \uctheucnum \label{ucInput}:] 
The max number of characters encoded is not likely to change, as our choice was based more on the allowances of the URL standard, rather than the allowances of any QR Code formats.

\end{description}

\section{Module Hierarchy} \label{SecMH}


\begin{description}
\item [\refstepcounter{mnum} \mthemnum \label{mHH}:]
\item ...
\end{description}


\begin{table}[h!]
\centering
\begin{tabular}{p{0.3\textwidth} p{0.6\textwidth}}
\toprule
\textbf{Level 1} & \textbf{Level 2}\\
\midrule

{Hardware-Hiding Module} & ~ \\
\midrule

\multirow{7}{0.3\textwidth}{Behaviour-Hiding Module} & ?\\
& ?\\
& ?\\
& ?\\
& ?\\
& ?\\
& ?\\ 
& ?\\
\midrule

\multirow{3}{0.3\textwidth}{Software Decision Module} & {?}\\
& ?\\
& ?\\
\bottomrule

\end{tabular}
\caption{Module Hierarchy}
\label{TblMH}
\end{table}

\section{Connection Between Requirements and Design} \label{SecConnection}

\paragraph{}
We dedicated a module aptly named GUI to cover and contain R3 entirely. 
We also made the input for the entire program simultaneous to combine couplings
with the new GUI (which was not part of the original project). We also 
decided to fill R4 through user input of desired size and error correction level.
This general interface will also cover R9, and will comply with R7.

\paragraph{}
In the case of input rejection we decided to fulfill the input check as soon as it is 
recieved from the GUI. To do this we simply layed the light check-code in the main function.
This set of checks encompasses solutions to R5, R6, R8 and R15.

\paragraph{}
To achieve R10 and R14 we minimilized the system by removing the moving-image QR 
Codes, which made sense considering that our expected application is mostly printed 
graphics (since a QR Code could be replaced with a linked image on a web page and 
unclickable digital advertisements are not yet as common). This will not make any 
remaining creations faster but will certainly cut down on program size.

\section{Module Decomposition} \label{SecMD}


\subsection{Hardware Hiding Modules (\mref{mHH})}

\begin{description}
\item[Secrets:]
\item[Services:]
\item[Implemented By:]
\end{description}

\subsection{Behaviour-Hiding Module}

\begin{description}
\item[Secrets:]
\item[Services:]
\item[Implemented By:] --
\end{description}

\subsubsection{Input Format Module (\mref{mInput})}

\begin{description}
\item[Secrets:]
\item[Services:]
\item[Implemented By:] 
\end{description}

\subsubsection{Etc.}


\subsection{Software Decision Module}

\begin{description}
\item[Secrets:] 
  
\item[Services:]  
  % Changes in these modules are more likely to be motivated by a desire to
  % improve performance than by externally imposed changes.
\item[Implemented By:] --
\end{description}

\subsubsection{Etc.}

\section{Traceability Matrix} \label{SecTM}


% the table should use mref, the requirements should be named, use something
% like fref
\begin{table}[H]
\centering
\begin{tabular}{p{0.2\textwidth} p{0.6\textwidth}}
\toprule
\textbf{Req.} & \textbf{Modules}\\
\midrule
R1 & \mref{mHH}, \mref{mInput}, \mref{mParams}, \mref{mControl}\\
R2 & \mref{mInput}, \mref{mParams}\\
R3 & \mref{mVerify}\\
R4 & \mref{mOutput}, \mref{mControl}\\
R5 & \mref{mOutput}, \mref{mODEs}, \mref{mControl}, \mref{mSeqDS}, \mref{mSolver}, \mref{mPlot}\\
R6 & \mref{mOutput}, \mref{mODEs}, \mref{mControl}, \mref{mSeqDS}, \mref{mSolver}, \mref{mPlot}\\
R7 & \mref{mOutput}, \mref{mEnergy}, \mref{mControl}, \mref{mSeqDS}, \mref{mPlot}\\
R8 & \mref{mOutput}, \mref{mEnergy}, \mref{mControl}, \mref{mSeqDS}, \mref{mPlot}\\
R9 & \mref{mVerifyOut}\\
R10 & \mref{mOutput}, \mref{mODEs}, \mref{mControl}\\
R11 & \mref{mOutput}, \mref{mODEs}, \mref{mEnergy}, \mref{mControl}\\
\bottomrule
\end{tabular}
\caption{Trace Between Requirements and Modules}
\label{TblRT}
\end{table}

\begin{table}[H]
\centering
\begin{tabular}{p{0.2\textwidth} p{0.6\textwidth}}
\toprule
\textbf{AC} & \textbf{Modules}\\
\midrule
\acref{acHardware} & \mref{mHH}\\
\acref{acInput} & \mref{mInput}\\
\acref{acParams} & \mref{mParams}\\
\acref{acVerify} & \mref{mVerify}\\
\acref{acOutput} & \mref{mOutput}\\
\acref{acVerifyOut} & \mref{mVerifyOut}\\
\acref{acODEs} & \mref{mODEs}\\
\acref{acEnergy} & \mref{mEnergy}\\
\acref{acControl} & \mref{mControl}\\
\acref{acSeqDS} & \mref{mSeqDS}\\
\acref{acSolver} & \mref{mSolver}\\
\acref{acPlot} & \mref{mPlot}\\
\bottomrule
\end{tabular}
\caption{Trace Between Anticipated Changes and Modules}
\label{TblACT}
\end{table}

\section{Use Hierarchy Between Modules} \label{SecUse}


\begin{figure}[H]
\centering
%\includegraphics[width=0.7\textwidth]{UsesHierarchy.png}
\caption{Use hierarchy among modules}
\label{FigUH}
\end{figure}

%\section*{References}

\bibliographystyle {plainnat}
\bibliography {MG}

\end{document}