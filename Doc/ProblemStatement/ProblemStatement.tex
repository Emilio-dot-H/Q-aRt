\documentclass{article}

\usepackage{tabularx}
\usepackage{booktabs}

\title{SE 3XA3: Problem Statement\\Q-aRt Code}

\author{Team \#, Q-aRt QTs
		\\ Elton Schiott schiotek
		\\ Emilio Hajj
		\\ Liam Duncan
}

\date{}

\begin{document}

\begin{table}[hp]
\caption{Revision History} \label{TblRevisionHistory}
\begin{tabularx}{\textwidth}{llX}
\toprule
\textbf{Date} & \textbf{Developer(s)} & \textbf{Change}\\
\midrule
2016-09-28 & Elton Schiott & Created LaTeX file, and edited text for clarity.\\
... & ... & ...\\
\bottomrule
\end{tabularx}
\end{table}

\newpage

\maketitle

\section{Problem Case}
	
	\paragraph{}
	
		The following project will focus on redeveloping a python based QR 
		encoder by enhancing and explanding certain features of the program. 
		The plan is to improve the scattered and ill-phrased documentation, especially the MIS, as well as refine 
		the algorithmic implimentation to handle jpg, png and gif files with greater simplicity.
	
\section{Stakeholders}

	\paragraph{}
	
		The stakeholders for this project are business and marketing firms that 
		wil use QR codes to advertise their products and services. Another 
		group of stakeholders is future developers who will use and modify the 
		application.
	
\section{Use Cases}

	\paragraph{}
	
		The QR code can be utilized as a marketing tool for major companies by 
		attracting users to webpages using artistic and unique Q-aRt generated 
		images. It can also be used by organizations for fundraising and 
		awareness purposes. It is a unique way to promote a business, product, 
		or event, and unlike the traditional black and white QR code, this 
		method will increase popularity due to its attractive visuals.

\end{document}