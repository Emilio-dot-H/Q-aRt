\documentclass{article}

\usepackage{booktabs}
\usepackage{tabularx}
\usepackage{hyperref}
\hypersetup{colorlinks=true}

\title{SE 3XA3: Development Plan\\Title of Project}

\author{Team 05, Q-aRt QTs
		\\ Elton Schiott schiotek
		\\ Emilio Hajj hajje
		\\ Liam Duncan duncanla
}

\date{}

\begin{document}

\begin{table}[hp]
\caption{Revision History} \label{TblRevisionHistory}
\begin{tabularx}{\textwidth}{llX}
\toprule
\textbf{Date} & \textbf{Developer(s)} & \textbf{Change}\\
\midrule
2016-09-30 & Elton Schiott & First revision of document\\

\bottomrule
\end{tabularx}
\end{table}

\newpage

\maketitle

This is the development plan for the project.

\section{Team Meeting Plan}
	
	\paragraph{}
		
		The team will meet on Tuesdays and Fridays for approximately one hour 
		at 2:30 PM. These meetings will occur regularly every week on the first 
		floor of Thode library. A meeting chair will be assigned ahead of time 
		for each meeting in order to keep discussions on track and make sure 
		all issues are addressed. A few rules will be established such as follows: a consensual approval of minutes (longevity of meeting), a summary report from each member, specific task assignments based on skill, a recap of unfinished business and an open-discussion of new problems to entail. 

\section{Team Communication Plan}
	
	\paragraph{}
	
		Tasks will be assigned in meetings. Other than in meetings the team 
		will communicate mainly through 2 means. The first is Git issue 
		tracking, which will be used to communicate the status of documents and 
		files that are currently being worked on. Other communication, such as 
		organizing meetings, changing plans, and more urgent messages will be 
		relayed through a group on Facebook.

\section{Team Member Roles}

	\paragraph{}
	
		There is no team leader assigned for the group, however one team member 
		will be assigned the role of chair for each meeting and this 
		responsibility will cycle through the group members. Notes will be 
		taken at each meeting by the chair.
		
	\paragraph{}
		
		Elton Schiott will take the lead role in documentation using LaTeX and 
		review of past milestones. Emilio Hajj will maintain the Gantt chart 
		and make sure tasks are completed on time. Liam Duncan will focus on 
		the implementation of the project. However, each member will be 
		assigned tasks and contribute to all areas of the project.
		
\section{Git Workflow Plan}

	\paragraph{}
		
		Git centralized workflow was selected as the method for the development of the project. Issue tracking will be used for task distribution and control. Labels will be applied to the issues as well as for completed documents to signify priority, category or anything else that can be found useful.
		
\section{Proof of Concept Demonstration Plan}

	\paragraph{}
		
		The proof of concept will be a simple implementation of a QR encoder. 
		The input will be an alphanumeric string and the program will generate 
		a standard QR code that, when scanned will read as the input string. 
		This is the most difficult part of the implementation, as the artistic 
		representations of QR codes will be generated using a similar method, 
		and the process has few neccesary inputs and functions, thus will not be difficult to 
		implement. 
		
	\paragraph{}
	
		The main risk in the redevelopment of this project is that the 
		algorithms used to generate the QR code are difficult to understand and 
		implement. If the proof of concept is able to successfully generate 
		correct, functional QR codes, this will demonstrate the required 
		knowledge to fully develop the encoder. This will include the artistic 
		QR code generator functionality. Testing may also be a concern, since 
		it will not be easy to unit test the QR codes that are generated for 
		correctness. Comparing the binary strings generated by the group's 
		implementations may be tested against those created by the original 
		implementation or other QR code encoding software to verify that the 
		software functions correctly.

\section{Technology}

	\paragraph{}
		
		The programming language to be used for the development of the software 
		will be Python as the original project uses this language and all group 
		members are comfortable with its use. The IDE of choice will be Geany 
		and documentation generation will be done using Sphinx. In regard of testing, various mobile applications will be used to verify the QR output (primarily Snapchat and 'QR Code Reader'). Unittest will be used for framework testing.  

\section{Coding Style}

	\paragraph{}
	
		\href{https://google.github.io/styleguide/pyguide.html}{Google's Python 
		style guide} will be used for any Python code in the project.
	
\section{Project Schedule}

	\paragraph{}
		
		Details about the project can be found in the 
		\href{https://gitlab.cas.mcmaster.ca/schiotek/Q-aRt_Code/tree/master/ProjectSchedule/GanttChartQ.gan}{Gantt
		 chart.}
		
\section{Project Review}
    \paragraph{}
        This project was set in order to re-implement an open source QR Code Generator making it more user-friendly and compatible. In terms of requirements for the design specifications and the overall set goals throughout this project, we as a team feel like we've accomplished and learned a lot throughout this term. We had a few fall-outs and problems committing in the very beginning. Partly caused by lack of issue tracking, focus and communication. We've grown a lot in terms of time management and commitment as a group and as individuals making this project a very valuable lesson. For projects to come, we hope to carry these new-found qualities and take on bigger and more complicated milestones.
        
        Concerning the technical skills, we come out of this project as better and more proficient coders in python as well as finer understating of QR codes, character encoding and data transmission.

\end{document}