\documentclass[12pt, titlepage]{article}

\usepackage{booktabs}
\usepackage{tabularx}
\usepackage{hyperref}
\hypersetup{
    colorlinks,
    citecolor=black,
    filecolor=black,
    linkcolor=red,
    urlcolor=blue
}
\usepackage[round]{natbib}

\title{SE 3XA3: Software Requirements Specification\\Q-aRt Code}

\author{Team 05, Q-aRt QTs
		\\ Elton Schiott schiotek
		\\ Emilio Hajj hajje
		\\ Liam Duncan duncanla
}

\date{\today}


\begin{document}

\maketitle

\pagenumbering{roman}
\tableofcontents
\listoftables
\listoffigures

\begin{table}[bp]
\caption{\bf Revision History}
\begin{tabularx}{\textwidth}{p{3cm}p{2cm}X}
\toprule {\bf Date} & {\bf Version} & {\bf Notes}\\
\midrule
2016/10/11 & 1.0 & Revision 0\\
2016/11/09 & 1.1 & Revision 1\\
\bottomrule
\end{tabularx}
\end{table}

\newpage

\pagenumbering{arabic}

This document describes the requirements for artistic QR Code generating 
software.  The template for the Software
Requirements Specification (SRS) is a subset of the Volere
template~\citep{RobertsonAndRobertson2012}.  

\section{Project Drivers}

\subsection{The Purpose of the Project}

	\paragraph{}
		
		QR Code stamps are not popular, even though they were a source of much 
		excitement when it was conceived that they could be used to more 
		conveniently and attractively direct smartphone holders to internet 
		addresses. QR codes are very practical but do take up much more space 
		that a written domain and may ruin the balance of a graphic 
		advertisement or notice. QR Codes may be more attractive to graphic 
		designers if they could easily and attractively incorporate logos and 
		colours, or additional information like the domain text itself to add 
		multiple functions to one graphical article, through a simple tool that 
		those without technical expertise could use. This open source project 
		is an exciting chance to explore and contribute to new tools that could 
		make this great technology more enticing and practical.

\subsection{The Stakeholders}
	
\subsubsection{The Client}

	Professor Spencer Smith is the client for this project. %capitalize the first word in a sentence
	the development for 
	the project will proceed as per the requirements and constraints specified 
	in the 3XA3 course.

\subsubsection{The Customers}

	Graphic designers are the preliminary target users of this product. 
	They should be able to apply the elements output by this program to 
	condense functions in their graphics and creatively present articles of 
	the work with a practical twist if this project is successful.

\subsubsection{Other Stakeholders}

	Those who view these graphics should be more likely to engage in the 
	intention of the graphics.
	
	Those who commission the graphic (whether they themselves be the 
	designer) should see a higher rate of views engaging their intentions 
	for the graphics.

\subsection{Mandated Constraints}

	The system should be an executable desktop program. It should run on a (at 
	least one) Windows 10 computer. Other systems are not required but are 
	expected to be applicable. The system must be completed by December 2016.

\subsection{Naming Conventions and Terminology}

		\paragraph{QR Code}
		
			 Similar to a barcode, a machine readable array of black 
			and white squares typically used to store URLs.
			\paragraph{URL}
			 Uniform Resource Locator. An address to a resource on the 
			internet.
			\paragraph{Version}
			 The size of a QR code, version 1 being 21 by 21 pixels, up 
			to version 40 which has 177 by 177 pixels.
			\paragraph{Error Correction Level}
			 A process that encodes bytes into the QR 
			code that allow the reader to determine if the data was read 
			correctly, and can be used to correct any errors that prevented it 
			from being read correctly.

\subsection{Relevant Facts and Assumptions}

	It is assumed that the user will have basic skills in the use of computers 
	especially in regards to file browsers.

\section{Functional Requirements}

\subsection{The Scope of the Work and the Product}

\subsubsection{The Context of the Work}
	
	The majority of any population that is regarded a target market of any 
	particular advertising campaign has a smartphone with the capability to 
	decode QR codes. This allows for those interested in an advertisement to 
	immediately learn more about the advertised product of service, increasing 
	the potential effectiveness of the advertisement.
	
\subsubsection{Work Partitioning}

	The partitioning of the work on the project is illustrated in the group's 
	Gantt chart.

\subsubsection{Individual Product Use Cases}

	\paragraph{}
	A graphic designer has a logo on a poster, and wants to make it the only 
	focus, but he is also tasked to put a link on the poster. Instead he takes 
	the logo and puts it into a qr code, and makes that the centre of the 
	poster.
	\paragraph{}
	 A business owner is making a sign and wants to link to their website, but 
	 doesn't want to have more text on the sign to keep it minimalistic and 
	 doesn't want to disrupt the colour, so he uses the system to make a qr 
	 code of the stock colour so it blends with the sign and maintains style.
	

\subsection{Functional Requirements}

	\paragraph{R1}
	System must accept image files of the format JPEG or PNG.
	\paragraph{R2}
	System must output a Portable Network Graphic (.png) file. No other file 
	formats are acceptable. 
	\paragraph{R3}
	System must have a GUI allowing the user to input an ascii string to 
	encode into a QR Code, and allowing the input of an image or no image if 
	desired to create an artistic QR Code or a standard QR Code if no image.
	\paragraph{R4}
	System must generate an appropriately versioned QR code based on the number of 
	characters in the user input text string (versions 1 to 40) and error 
	correction level (L,M,Q, or H).
	\paragraph{R5}
	System must not allow the user to enter a string that contains unsupported 
	characters for the alphanumeric mode of QR encoding, and must prevent the 
	user from entering a string of more than MAXIMUM characters.
	\paragraph{R6}	
	When a non compatible file is chosen, there must be a visible warning. The 
	warning explains the allowed formats and only allows acceptance. The 
	acceptance brings the GUI back to the original “browse for a file” state.
	
	

\section{Non-functional Requirements}

\subsection{Look and Feel Requirements}
	
	The system must have a simple, intuitive interface.
	
	
\subsection{Usability and Humanity Requirements}

\paragraph{R7}
	The system must notify the user when an invalid input is entered for either 
	the text or image.
	\paragraph{R8}
	the system must prompt the user for an appropriate input in the GUI, to 
	remove confusion about the requested input for the field.

\subsection{Performance Requirements}

	\paragraph{R9}
	The system will output the specified QR code within 5 seconds.
	\paragraph{R10}
	The output of the system must be recognizable as the original images. A 
	viewer must be able to distinguish the QR code image that represents one 
	input image from another QR code image which represents an input that is 
	different. Here	different refers to a significant difference in colour 
	palate, distribution of high activity areas in the input image, and text 
	which differs by more than two letters and significantly in format. Here 
	high activity refers to	a distinct logo, or a clearly visible object or 
	subject in that area.

\subsection{Operational and Environmental Requirements}
	
	\paragraph{R11}
	The system will be used on a desktop or laptop computer.
	\paragraph{R12}
	The system will be used with a Python interpreter.
	
\subsection{Maintainability and Support Requirements}

	\paragraph{R13}
	The system will be small enough to allow for easy replacement of the system 
	with updated versions.
	\paragraph{R14}
	The system will produce meaningful messages when the user inputs incorrect 
	input to allow the user to resolve issues.

\subsection{Security Requirements}

	None.

\subsection{Cultural Requirements}

	None. Any images used to generated QR codes are the responsibility of the 
	user.

\subsection{Legal Requirements}

	\paragraph{R15}
	The development of the system will follow the guidelines of GNU General 
	Public License Version 3.

\subsection{Health and Safety Requirements}

	\paragraph{R16}
	The system must not cause any critical failure of the computer it is run on 
	that may pose a risk to the health and safety of the user.

\section{Project Issues}

\subsection{Open Issues}

	The process of encoding a QR code from an alphanumeric string has not been 
	fully researched by the group. This may slow down the implementation of the 
	software. The software also requires the use of a few Python libraries 
	which the group is not entirely familiar with at this point.

\subsection{Off-the-Shelf Solutions}

	Documentation on the encoding process is readily available, as are numerous 
	open source examples of QR encoders. These will make it possible to attain 
	the required knowledge to complete the implementation. The libraries will 
	also be useful for the implementation of the software as they allow 
	components such as image IO not to be redeveloped as they are basic and not 
	part of the main purpose of the software.

\subsection{New Problems}

	None.

\subsection{Tasks}

	At this stage, a proof of concept demo is in development that will create 
	one size of QR code from an alphanumeric string without the artistic 
	component. In parallel to this, a testing plan, and a Gantt chart are being 
	constructed and modified. All these and this requirements document will 
	continue to be refined throughout the development process.

\subsection{Migration to the New Product}

	This project is a redevelopment of a current existing product, and is a 
	portable solution. As such, there are no issues in migration to the new 
	product.

\subsection{Risks}

	As the project is mostly centered around one algorithm for the generation 
	of QR codes, and image quality is an important requirement for the output 
	of the system, there are risks involving difficulty in testing the product. 

\subsection{Costs}

	The product may not meet requirements adequately if components of the 
	software are unable to be tested well. This will result in an incomplete 
	solution for the project, as there may be many untested cases where the 
	software fails.

\subsection{User Documentation and Training}

	The software as intended will not require extensive training if any, as the 
	only knowledge required to use the software is basic knowledge of the file 
	explorer to select input images, as well as to retrieve generated QR codes, 
	and basic knowledge on URLs. Documentation is an important part of the 
	redevelopment of this project and as such there will be an ample amount of 
	documentation that will be available to the user if required.

\subsection{Waiting Room}
	
	N/A
	
\subsection{Ideas for Solutions}

	The algorithm must be understood well and modularized to allow for unit 
	testing, and the requirements for acceptable image quality must be 
	well-defined before testing.

\bibliographystyle{plainnat}

\bibliography{SRS}

\newpage

\section{Appendix}


\subsection{Symbolic Parameters}

MAXIMUM - 4296 (the maximum capacity of the largest QR code, version 40-L)


\end{document}