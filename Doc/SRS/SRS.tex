\documentclass[12pt, titlepage]{article}

\usepackage{booktabs}
\usepackage{tabularx}
\usepackage{hyperref}
\hypersetup{
    colorlinks,
    citecolor=black,
    filecolor=black,
    linkcolor=red,
    urlcolor=blue
}
\usepackage[round]{natbib}

\title{SE 3XA3: Software Requirements Specification\\Title of Project}

\author{Team 05, Q-aRt QTs
		\\ Elton Schiott schiotek
		\\ Emilio Hajj hajje
		\\ Liam Duncan duncanla
}

\date{\today}


\begin{document}

\maketitle

\pagenumbering{roman}
\tableofcontents
\listoftables
\listoffigures

\begin{table}[bp]
\caption{\bf Revision History}
\begin{tabularx}{\textwidth}{p{3cm}p{2cm}X}
\toprule {\bf Date} & {\bf Version} & {\bf Notes}\\
\midrule
Date 1 & 1.0 & Notes\\
Date 2 & 1.1 & Notes\\
\bottomrule
\end{tabularx}
\end{table}

\newpage

\pagenumbering{arabic}

This document describes the requirements for artistic QR Code generating 
software.  The template for the Software
Requirements Specification (SRS) is a subset of the Volere
template~\citep{RobertsonAndRobertson2012}.  

\section{Project Drivers}

\subsection{The Purpose of the Project}

	\paragraph{}
		
		QR Code stamps are not popular, even though they were a source of much 
		excitement when it was conceived that they could be used to more 
		conveniently and attractively direct smartphone holders to internet 
		addresses. QR codes are very practical but do take up much more space 
		that a written domain and may ruin the balance of a graphic 
		advertisement or notice. QR Codes may be more attractive to graphic 
		designers if they could easily and attractively incorporate logos and 
		colours, or additional information like the domain text itself to add 
		multiple functions to one graphical article, through a simple tool that 
		those without technical expertise could use. This open source project 
		is an exciting chance to explore and contribute to new tools that could 
		make this great technology more enticing and practical.

\subsection{The Stakeholders}

	\paragraph{}
		Graphic designers are the preliminary target users of this product. 
		They should be able to apply the elements output by this program to 
		condense functions in their graphics and creatively present articles of 
		the work with a practical twist if this project is successful.
		
		Those who view these graphics should be more likely to engage in the 
		intention of the graphics.
		
		Those who commission the graphic (whether they themselves be the 
		designer) should see a higher rate of views engaging their intentions 
		for the graphics.
		
	
\subsubsection{The Client}

\subsubsection{The Customers}

\subsubsection{Other Stakeholders}

\subsection{Mandated Constraints}

	The system should be an executable desktop program. It should run on a (at 
	least one) Windows 10 computer. Other systems are not required but are 
	expected to be applicable.

\subsection{Naming Conventions and Terminology}

		\paragraph{}
		
			There is no special terminology expected to be used, any incidental 
			acronyms will be explained as they appear.

\subsection{Relevant Facts and Assumptions}

User characteristics should go under assumptions.

\section{Functional Requirements}

\subsection{The Scope of the Work and the Product}

\subsubsection{The Context of the Work}

\subsubsection{Work Partitioning}

\subsubsection{Individual Product Use Cases}

\subsection{Functional Requirements}

	System must output either a Portable Network Graphic (.png) or Graphics 
	Interchange Format (.gif) file. This will depend on whether the input is a 
	JPEG or PNG file (output PNG) or a GIF file (output GIF). No other file 
	formats will be accepted. 
	
	System must have a GUI allowing the user to input an alphanumeric string to 
	encode into a QR code, and allowing the input an image if desired to create 
	an artistic QR Code.
	
	System must generate an appropriate version QR code based on the number of 
	characters in the user input text string (versions 1 to 40) and error 
	correction level (L,M,Q, or H).
	
	System must not allow the user to enter a string that contains unsupported 
	characters for the alphanumeric mode of QR encoding, and must prevent the 
	user from entering a string of more than 4296 characters (the maximum 
	capacity of the largest QR code, version 40-L).
		
	When a non compatible file is chosen, there must be a visible warning. The 
	warning explains the allowed formats and only allows acceptance. The 
	acceptance brings the GUI back to the original “browse for a file” state.
	
	The output must be recognizable as the original images. A viewer must be 
	able to distinguish the QR code image that represents one input image from 
	another QR code image which represents an input that is different. Here 
	different refers to a significant difference in colour palate, distribution 
	of high activity areas in the input image, and text which differs by more 
	than two letters and significantly in format. Here high activity refers to 
	a distinct logo, or a clearly visible object or subject in that area.
	
	

\section{Non-functional Requirements}

\subsection{Look and Feel Requirements}

\subsection{Usability and Humanity Requirements}

\subsection{Performance Requirements}

\subsection{Operational and Environmental Requirements}

\subsection{Maintainability and Support Requirements}

\subsection{Security Requirements}

\subsection{Cultural Requirements}

\subsection{Legal Requirements}

\subsection{Health and Safety Requirements}

This section is not in the original Volere template, but health and safety are
issues that should be considered for every engineering project.

\section{Project Issues}

\subsection{Open Issues}

\subsection{Off-the-Shelf Solutions}

\subsection{New Problems}

\subsection{Tasks}

\subsection{Migration to the New Product}

\subsection{Risks}

\subsection{Costs}

\subsection{User Documentation and Training}

\subsection{Waiting Room}

\subsection{Ideas for Solutions}

\bibliographystyle{plainnat}

\bibliography{SRS}

\newpage

\section{Appendix}

This section has been added to the Volere template.  This is where you can place
additional information.

\subsection{Symbolic Parameters}

The definition of the requirements will likely call for SYMBOLIC\_CONSTANTS.
Their values are defined in this section for easy maintenance.


\end{document}